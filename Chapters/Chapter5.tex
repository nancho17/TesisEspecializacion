% Chapter Template

\chapter{Conclusiones} % Main chapter title

\label{Chapter5} % Change X to a consecutive number; for referencing this chapter elsewhere, use \ref{ChapterX}


%----------------------------------------------------------------------------------------

%----------------------------------------------------------------------------------------
%	SECTION 1
%----------------------------------------------------------------------------------------

\section{Conclusiones generales }

La idea de esta sección es resaltar cuáles son los principales aportes del trabajo realizado y cómo se podría continuar. Debe ser especialmente breve y concisa. Es buena idea usar un listado para enumerar los logros obtenidos.

El logro de este trabajo fue pasar de la idea de un dispositivo a su diseño y fabricación ,

Poder crear desde la idea demuestra la posibilidad de elaborar dispositivos electrónicos para necesidades de la industria de manera local.

Se cumplieron todos los requisitos planteados al comienzo del trabajo pero bien, durante la elaboración se encontraron otras necesidades que deberán ser cumplidas en una actualización del software.

Se realizo un manual de usuario para poder comercializar el dispositivo, el manual incluye las limitaciones y alcances del dispositivo como también especificaciones de como usarlo.

El trabajo brindo muchísima experiencia en como se desarrolla la electrónica se pudo ver los diferentes puntos de vista en la manera de hacerlo tanto del sector académico como del privado.

En el desarrollo de este trabajo se aplicaron conocimientos obtenidos de las materias de la \textbf{Carrera de Especialización en Sistemas Embebidos}. Podemos resaltar:

\begin{itemize}
\item Técnicas de ingeniería de software y uso de repositorios para mantener organizadas las versiones tanto del software como del hardware.
\item Diseño de PCB, las instrucciones y consejos de como realizar un buen diseño.
\item Fundamentos en la manufactura electrónica fue clave para la fabricación del pcb.
\item Los conocimientos de programación y arquitectura de microcontroladores facilitaron la curva de aprendizaje del microcontrolador usado.
\item Los conocimientos de protocolos y comunicaciones digitales al implementar en el software protocolos comúnmente conocidos y protocolos nuevos.
\end{itemize}

%----------------------------------------------------------------------------------------
%	SECTION 2
%----------------------------------------------------------------------------------------
\section{Próximos pasos}

El trabajo elaborado consideraba que en el dispositivo debía implementarse un puerto LAN ethernet en un futuro y durante el desarrollo del trabajo se encontraron otras necesidades que debían completarse:

\begin{itemize}
\item Implementar en el software un driver previamente desarrollado para el puerto LAN ethernet.
\item Implementar un menú de calibración del dispositivo para que el fabricante pueda producirlo en masa.
\item Agregar todos los detalles pertinentes al dispositivo en el manual de usuario. Elaborar la segunda version del manual de usuario.
\item También agregaría valor al trabajo elaborar una aplicación, tipo web app o mas compleja, para comunicarse con el dispositivo.
\end{itemize}







