% Chapter Template

\chapter{Conclusiones} % Main chapter title

\label{Chapter5} % Change X to a consecutive number; for referencing this chapter elsewhere, use \ref{ChapterX}


%----------------------------------------------------------------------------------------

%----------------------------------------------------------------------------------------
%	SECTION 1
%----------------------------------------------------------------------------------------

\section{Conclusiones generales }
Del trabajo realizado se logró obtener un dispositivo medidor de potencia eléctrica con tolerancias aceptables, capaz de transmitir datos en un entorno industrial.

El trabajo demostró que a partir de requerimientos establecidos por un cliente se puede diseñar la electrónica necesaria para cumplirlos. También se demuestra que el diseño y programación de un prototipo electrónico puede llevarse a cabo de manera completamente remota.

Se cumplieron los requisitos planteados al comienzo del trabajo. Durante la elaboración se encontraron otras necesidades que deberán ser satisfechas en una actualización del software.

Se realizó un manual de usuario para poder comercializar el dispositivo. El manual incluye las limitaciones del dispositivo como también especificaciones de cómo usarlo.

En el desarrollo de este trabajo se aplicaron conocimientos obtenidos de las materias de la Carrera de Especialización en Sistemas Embebidos. Podemos resaltar:

\begin{itemize}
\item Técnicas de ingeniería de software y uso de repositorios para mantener organizadas las versiones tanto del software como del hardware.
\item De la materia \textquotedblleft Diseño de PCB\textquotedblright \  los métodos para nombrar y realizar esquemas prolijos en circuitos de varias hojas.
\item Los fundamentos de la materia \textquotedblleft Diseño para la manufactura electrónica\textquotedblright \  fueron claves para la fabricación del PCB.
\item Los conocimientos de programación y arquitectura de microcontroladores facilitaron la curva de aprendizaje del microcontrolador usado.
\item Los conocimientos de protocolos y comunicaciones digitales al implementar en el software protocolos comúnmente conocidos y protocolos nuevos.
\end{itemize}

%----------------------------------------------------------------------------------------
%	SECTION 2
%----------------------------------------------------------------------------------------
\section{Próximos pasos}

El trabajo elaborado consideraba que en el dispositivo debía implementarse un puerto LAN Ethernet en un futuro. Durante el desarrollo del trabajo se encontraron posibles evoluciones del proyecto. Las siguientes tareas a realizar son:

\begin{itemize}
\item Implementar en el software un driver previamente desarrollado para el puerto LAN Ethernet.
%\item Implementar un menú de calibración del dispositivo para que el fabricante pueda producirlo en masa.
%\item Agregar todos los detalles pertinentes al dispositivo en el manual de usuario. Elaborar la segunda version del manual de usuario.
\item También agregaría valor al trabajo elaborar una aplicación, tipo web app o más compleja, para agregar un método adicional de configurar el dispositivo como también visualizar los datos que este comunica.
\end{itemize}




